\begin{exo}[Une Intégrale de Frullani]
	\label{intgen1}
	On pose pour $x\in\R$, $f(x)=\int_0^{+\infty}\frac{\arctan(xt)-\arctan(t)}t\dd t$
	\begin{enumerate}
		\item Déterminer $\mathcal D_f$ domaine de définition de $f$.
		\item Déterminer le domaine de classe $\mathcal C^1$ de $f$.
		\item En déduire une expression de $f(x)$ pour $x\in\mathcal D_f$.
		\item Retrouver le résultat de la question (iii) sans utiliser le théorème de dérivation des intégrales à paramètre\\
	\end{enumerate}
\end{exo}

\begin{exo}[Calcul d'équivalent 1]
	\label{intgen2}
	$I(x)=\int_0^1\frac{t^x}{1+t}\dd t$.
	\begin{enumerate}
		\item Domaine de définition de $I$ ?
		\item Calculer $I(x) + I(x + 1)$.
		\item Équivalent de $I(x)$ en $+\infty$ ?
	\end{enumerate}
\end{exo}

\begin{exo}[Calcul d'équivalent 2]
	\label{intgen3}
	$F(x) = \int_0^1\frac{\sin(tx)}{1+t}\dd t$. Montrer que $F$ est définie et continue sur $\Rpe$. Montrer que $F(x)\sim_0\frac\pi{2x}$.
\end{exo}


