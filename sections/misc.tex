\begin{exo}[Cardinal maximal d'une partie fade]
    \label{misc1}
    Une partie $A$ de $\N$ est dite fade si pour tous $x,y\in A$, $x+y\notin A$.
    Calculer le cardinal maximal d'une partie fade incluse dans $\inl1n$ pour $n\in\N^*$.
\end{exo}

\begin{exo}[Dérivée seconde]
    \label{misc2}
    Soit $f:[a,b]\to\R$ une fonction continue avec $a<b$. 
    Lorsque la limite existe, on note $\Delta f(x)$ la quantité 
    \[
        \lim_{h\to 0}\frac{f(x+h)+f(x-h)-2f(x)}{h^2}
    \]
    \begin{enumerate}
        \item Si $f$ est de classe $\mathcal C^2$, montrer que $\Delta f$ est bien définie sur $(a,b)$ et est continue.
        \item Si $\Delta f$ est bien définie et nulle sur $(a,b)$, montrer que $f$ est affine.
        \item Si $\Delta f$ est bien définie et continue sur $(a,b)$, montrer que $f$ y est $\mathcal C^2$. 
    \end{enumerate}
\end{exo}