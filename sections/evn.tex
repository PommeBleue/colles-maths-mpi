\begin{exo}[CNS pour qu'un sous-groupe de $\C^*$ soit fermé]
    \label{evn1}
    Donner des conditions nécessaires et suffisantes 
    sur $z\in\C$ pour que $G_z=\lbrace e^{itz},\ t\in\Z\rbrace$
    soit un sous-groupe fermé de $\C^*$.
\end{exo}

\begin{exo}[Parties de $\gl_n(\R)$ compactes, non vides et stables par produit]
    \label{evn2}
    Soit $X$ une partie de $\gl_n(\R)$ non vide, compacte et stable par produit. Montrer que $X$ est un sous-groupe de $\gl_n(\R)$.
\end{exo}
    
\begin{exo}[L'ensemble des polynômes unitaires scindés est un fermé]
    \label{evn3}
    On munit $\R[X]$ de la norme $\norm{\cdot}_{\infty}$ définie par $\norm{\sum_{k=0}^{+\infty}a_kX^k}=\text{max}\lbrace |a_k|,\ k\in\N\rbrace$.
    \begin{enumerate}
        \item Montrer que $\mathcal U$ l'ensemble des polynômes unitaires de $\R[X]$ est fermé.
        \item Soit $Q\in\mathcal U$ non constant, on note $p=\deg Q$. Montrer que 
        \[Q\text{ est scindé sur }\R\iff \forall z\in\C,\ |Q(z)|\geq |\Im(z)|^p\]
        \item Montrer que $\mathcal S$, l'ensemble des polynômes unitaires et scindés de $\R[X]$ est un fermé.
    \end{enumerate}
\end{exo}

\begin{exo}[Somme d'une partie fermée et d'une partie compacte]
    \label{evn4}
	Soient $E$ un espace vectoriel normé, $F$ une partie fermée de $E$ et $K$ une partie compacte de $E$. Montrer que $F+K$ est une partie fermée de $E$.
\end{exo}

\begin{exo}[Topologie du groupe orthogonal]
    \label{evn5}
	Soit $n\in\N^*$. On rappelle que le groupe orthogonal est défini par \[O_n(\R):=\lbrace M\in\mathcal M_n(\R)\ |\ {}^tMM=I_n\rbrace\]
	Cet ensemble est-il fermé dans $\mathcal M_n(\R)$ ? Est-il connexe par arcs ?
\end{exo}

\begin{exo}[Fonctions injectives de ${[0,1]}^2$ dans $\R$]
    \label{evn6}
    Existe-t-il des fonctions continues injectives de $[0,1]^2$ dans $\R$ ?
\end{exo}