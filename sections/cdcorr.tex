\begin{proof}[Correction de l'exercice \ref{cd1}]
    Soit $y_0\in U$. 
    Il existe $\varepsilon >0$ tel que $V=B(y_0,\varepsilon)\subseteq U$. 
    On prend alors $x,y\in V$ et on pose $\varphi:t\in[0,1]\mapsto f(x+t(y-x)) \in \mathcal C^1$ par composition,
    de dérivée $\varphi'(t)=\dd f_{\varphi(t)}(y-x)$. 
    On a alors
    \[
        \forall t\in[0,1],\ \norm{\dd f_{\varphi(t)}(y-x)}\leq\norm{\dd f_{\varphi(t)}}_{\mathcal L(\R^n,\R^n)}\norm{y-x}
    \]
    et $\dd f$ étant continue, et $\varphi$ étant à valeurs dans un compact, $\norm{\dd f_{\varphi(t)}}_{\mathcal L(\R^n,\R^n)}$ 
    est majorée par une constante $L_{y_0}$ sur $[0,1]$.
    On a ensuite 
    \begin{align*}
        \norm{f(y)-f(x)}\leq\norm{\varphi(1)-\varphi(0)}&=\norm{\int_0^1\varphi'(t)\dd t}\\ &\leq\int_0^1\norm{\varphi'(t)}\dd t\leq L_{y_0}\norm{y-x}
    \end{align*}
    et ceci vaut pour tous $x,y\in V$. 
    Cela achève de démontrer la thèse de l'énoncé.
\end{proof}

\begin{proof}[Correction de l'exercice \ref{cd2}]
    En effet, $B_f$ étant compact ($\R^n$ est de dimension finie, pardi) et $f$ étant continue sur $B_f$ à valeurs réelles,
    $f$ est bornée sur $B_f$ et on peut noter $f(x_0)=m=\min f$ et $f(x_1)=M=\max f$ (les min et max étant pris sur $B_f$).
    Deux cas de figure se présentent :
    \begin{itemize}
        \item Soit $m = M$, alors $f$ est constante sur tout $B_f$ et le résultat est trivial ;
        \item soit $m<M$, et nécessairement $x_1\notin \mathbf S^{n-1}$ ou $x_0\notin \mathbf S^{n-1}$, car $f$ y est constante. 
        Si par exemple $x_0\notin\mathbf S^{n-1}$, alors $x_0\in B$, donc $f$ admet un extremum local en $x_0$ sur l'ouvert $B$, et alors $\dd f_{x_0}=0$.
    \end{itemize}
\end{proof}