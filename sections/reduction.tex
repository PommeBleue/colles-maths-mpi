\begin{exo}[CNS valeur propre commune]
	\label{reduction1}
	Soient $A,B\in\mathcal M_n(\C)$. Montrer que les propositions suivantes sont équivalentes :
	\begin{enumerate}
		\item $A$ et $B$ possèdent une valeur propre commune.
		\item Il existe $M\in\mathcal M_n(\C)$ non nulle telle que $AM=MB$
		\item $\mu_A(B)\notin \gl_n(\C)$
	\end{enumerate}
\end{exo}

\begin{exo}[$P(A)$ diagonalisable et $P'(A)$ inversible $\implies$ $A$ diagonalisable]
	\label{reduction2}
    Soit $A\in\mathcal M_n(\C)$ et $P\in\C[X]$ tel que $P(A)$ est diagonalisable et $P'(A)$ est inversible. Montrer que $A$ est diagonisable.
\end{exo}

\begin{exo}[Diagonalisation dans $\mathcal M_n(\mathbb F_p)$]
	\label{reduction3}
    Soit $p$ un nombre premier, $n\in\N^*$ et $A\in\mathcal M_n(\mathbb F_p)$. Montrer que \[A\quad \text{diagonalisable}\quad \iff\quad A^p=A\]
\end{exo}

\begin{exo}[Matrice semblable à son double]
	\label{reduction4}
	Soit $M\in\mathcal M_n(\C)$ telle que $M\underset{\text{sb}}{\sim}2M$. Montrer que $M$ est nilpotente.
\end{exo}

\begin{exo}[Comparaison de polynômes minimaux]
	\label{reduction5}
    Soit $E$ un $K$ espace vectoriel, $f\in L(E)$ et $G:g\in L(E)\mapsto f\circ g$. Vérifier que $G\in L(L(E))$ et comparer (sous réserve d'existence) les polynômes minimaux de $f$ et $G$.
\end{exo}

\textit{Les deux exercices qui vont suivre n'ont jamais été posés lors de mon année scolaire, mais ont fait partie de mes réflexions lors de mon année de MPI*, j'ai alors tenu à les inclure lors de la première rédaction de ce document.}

\begin{exo}[Coefficients du polynôme caractéristique]
	\label{reduction6}
	Soit $M\in\mathcal M_n(\K)$. On appelle mineur principal d'ordre $k\in\lbrace1,\dots,n\rbrace$ le déterminant d'un $M_I=(m_{i,j})_{(i,j)\in I^2}$ avec $I\subset\lbrace 1,\dots,n\rbrace$ tel que $\Card(I)=k$. Donner une expression des coefficients de $\chi_M$ en fonction des mineurs principaux.
\end{exo}

\begin{exo}[Limite d'une suite de matrices]
	\label{reduction7}
	Soit $M\in\mathcal M_n(\K)$, $\K = \R$ ou $\C$. Donner une condition nécessaire et suffisante pour que la suite $(M^p)_{p\in\N}$ converge. Que dire sur la valeur de la limite en cas de convergence ?
\end{exo}