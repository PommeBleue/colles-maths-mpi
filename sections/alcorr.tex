\begin{proof}[Correction de l'exercice \ref{al1}]\hfill
    \begin{enumerate}
        \item $A\subset \Vect(A)$, donc par définition, 
        $\forall i,j,k,l\in\inl{1}{n},\ E_{ij}E_{kl}-E_{kl}E_{ij}=\delta_{jk}E_{il}-\delta_{li}E_{kj}\in A$. 
    
        Il s'ensuit que
        \[
            \forall i,l\in\inl1n,\ 
            \left\lbrace\begin{array}c i\neq l\implies E_{il}\in \Vect(A)\\
                 \forall i\in\inl{2}{n},\ E_{ii}-E_{11}\in \Vect(A)\end{array}\right. 
        \]
        On pose $\mathcal F=(E_{ij})_{i\neq j}\cup(E_{ii}-E_{11})_{i\in\inl2n}$.
        Cette famille est clairement libre et contient $n^2-1$ vecteurs de $\Vect(A)$,
        d'où $\dim\Vect(A)\geq n^2-1$. 
        Or $\Vect(A)\subset \ker\tr$, et $\tr$ étant une forme linéaire non nulle,
        on a $\dim\ker\tr=n^2-1$, donc $\dim\Vect(A)\leq n^2-1$ 
        donc $\dim\Vect(A)=n^2-1=\dim\ker\tr$ puis $\Vect(A)=\ker\tr$.
        \item De même que pour $\tr$, on a $\Vect(A)\subset \ker\varphi$
        donc $\ker\tr\subset \text{Ker}\varphi$, donc il existe $\lambda\in\R$
        tel que $\varphi=\lambda\text{tr}$. Mais $\varphi(I_n)=n=\text{tr}(I_n)$
        donc $n=\lambda n$ puis $\lambda = 1$ (sauf si $n=0$, mais ce cas est trivial).
    \end{enumerate}
\end{proof}

\begin{proof}[Correction de l'exercice \ref{al2}]
    Pour $i\in\inl1p$, on note $X_i=(x_i^{(1)},\dots,x_i^{(n)})$. On a alors, \[\forall i\in\inl1p,\ X_i{}^tX_i=(x_i^{(1)}X_i|\dots|x_i^{(n)}X_i)\]
    Soit $(\lambda_1,\dots,\lambda_p)\in\R^p$ telle que $\sum_{i=1}^p\lambda_iX_i{}^tX_i=0$.
    Or, $(X_1,\dots,X_p)$ formant une famille libre, aucun des vecteurs la constituant n'est nul d'où
    \[\forall i\in\inl1p,\ \exists l_i\in\inl1p,\ x_i^{(l_i)}\neq 0\]
    Ainsi, pour $i\in\inl1p$, en regardant que la $l_i$ème colonne dans la somme nulle écrite plus haut, 
    on a \[\sum_{j=1}^p\lambda_jx_j^{(l_i)}X_j=0\]
    $(X_1,\dots,X_p)$ étant libre, on a $\forall j\in\inl1p,\ \lambda_jx_j^{(l_i)}=0$, en particulier $\lambda_ix_i^{(l_i)}=0$
    donc $\lambda_i=0$ ($x_i^{l_i}\neq 0$). 
    Finalement, ceci étant vrai pour tout $i\in\inl1p$, on a 
    \[\forall i\in\inl1p,\ \lambda_i=0\] et $(X_1{}^tX_1,\dots,X_p{}^tX_p)$ est libre dans $\mathcal M_n(\R)$
    (il faut bien sûr préciser que la taille des matrices $X_i{}^tX_i$ est de $n\times n$, mais cela est bien clair).
\end{proof}